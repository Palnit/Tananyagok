\documentclass[12pt,a4paper]{article}
\usepackage[utf8]{inputenc}
\usepackage[magyar]{babel}
\usepackage[T1]{fontenc}
\usepackage{amsmath}
\usepackage{amsfonts}
\usepackage{amssymb}
\usepackage{graphicx}
\author{Petrányi Bálint}\usepackage[pdfusetitle]{hyperref}
\usepackage{titling}
\usepackage[x11names]{xcolor}
\usepackage{xparse}
\usepackage{tcolorbox}
\usepackage{graphicx}
\usepackage{enumerate}
\graphicspath{ {./images/} }

\hypersetup{
    colorlinks,
    citecolor=black,
    filecolor=black,
    linkcolor=black,
    urlcolor=black
}

\date{\today}
\author{Petrányi Bálint}
\title{%
	\textbf{Analzízis Alkalmazásai.} \\
	\textbf{Programtervező informatikus A. szakirány} \\
	RöpZh Tételek\\
	\large 2023-2024. tanév 2. félév
}
\newcommand{\norm}[1]{\lVert #1 \rVert}
\newcommand{\R}{\mathbb{R}}
\newcommand{\CR}{\mathcal{R}}
\newcommand{\N}{\mathbb{N}}
\newcommand{\CD}{\mathcal{D}}
\newcommand{\CL}{\mathcal{L}}
\newcommand{\m}{\varrho}
\newcommand{\f}{\varphi}
\newcommand{\met}[1]{\m \left( #1 \right)}
\newcommand{\bb}[1]{\left( #1 \right)}
\newcounter{count}
%\setcounter{count}{#3} #1_{\thecount},\setcounter{count}{\numexpr\thecount+1} #1_{\thecount}, \ldots, #2_{#4} 
\NewDocumentCommand{\ton}{ O{1} O{n} o o } {\IfValueTF{#3}{\IfValueTF{#4}{\setcounter{count}{#3} #1_{\thecount},\setcounter{count}{\numexpr\thecount+1} #1_{\thecount}, \ldots, #2_{#4}}{\setcounter{count}{#3} #1_{\thecount},\setcounter{count}{\numexpr\thecount+1} #1_{\thecount}, \ldots, #2_{n}}}{ #1, #1, \ldots, #2 }}
\newcommand{\oneton}{1,2,\ldots ,n}
\newcommand{\nrom}[1]{\lVert #1 \rVert}
\newcommand{\angels}[1]{\langle #1 \rangle}
\newcommand{\abs}[1]{\left| #1 \right|}
\newcommand{\braces}[1]{\left\lbrace #1 \right\rbrace}
\newcommand{\boxes}[1]{\left[ #1 \right]}


\newtheorem{tet}{Tétel}[section]
\tcolorboxenvironment{tet}{
  colback=blue!10!white,
  boxrule=0pt,
  boxsep=1pt,
  left=2pt,right=2pt,top=2pt,bottom=2pt,
  oversize=2pt,
  sharp corners,
  before skip=\topsep,
  after skip=\topsep,
}

\newtheorem{defi}{Definíció}[section]
\tcolorboxenvironment{defi}{
  colback=green!15!white,
  boxrule=0pt,
  boxsep=1pt,
  left=2pt,right=2pt,top=2pt,bottom=2pt,
  oversize=2pt,
  sharp corners,
  before skip=\topsep,
  after skip=\topsep,
}

\tcolorboxenvironment{proof}{
  colback=red!30!white,
  boxrule=0pt,
  boxsep=1pt,
  left=2pt,right=2pt,top=2pt,bottom=2pt,
  oversize=2pt,
  sharp corners,
  before skip=\topsep,
  after skip=\topsep,
}

\begin{document}
\maketitle
\tableofcontents
\newpage
\section{week}
\subsection{Mikor mondjuk, hogy a $\f$ függvény az $f(x, y) = 0$ egyenletnek egy implicit megoldása?}
\begin{defi}
Legyen $f \in \R^2 \rightarrow \R$ egy adott függvény. Ha létezik olyan $I \subset \R$ nyílt intervallum és $\f : I \rightarrow \R$ függvény, hogy 
\[
f(x,\f(x)) = 0 \quad (\forall x \in I)
\]
akkor azt mondjuk, hogy a $\f$ függvény az $f(x, y) = 0$ implicit alakban van megadva
\end{defi}
\subsection{Hogyan szól az egyváltozós implicitfüggvény-tétel?}
\begin{tet}[Egyváltozós implicitfüggvény-tétel.]
Legyen $\Omega \in \R^2$ nyílt halmaz és $f : \Omega \rightarrow \R$. Tegyük fel, hogy
\begin{enumerate}
\item[(a)] $f$  folytonosan deriválható $\Omega$-n
\item[(b)] az $(a,b) \in \Omega$ pontban $f(a,b) = 0 $ és $\partial_2 f(a,b) \neq 0$
\end{enumerate}
Ekkor:
\begin{enumerate}
\item Van olyan $K(a) =: U$ és $K(b) =: V$ környezet $\R$-ben, hogy minden $x \in U$ ponthoz létezik egyetlen $\f(x) \in V$, amelyre $f(x, \f(x)) = 0$
\item Az így definiált $\f : U \rightarrow V$ függvény folytonosan deriválható $U$-n, továbbá $\forall x \in U$-ra  $\partial_2 f(x,\f(x)) \neq 0$ és
\end{enumerate}
\[
\f'(x) = - \frac{\partial_1 f (x,\f(x))}{\partial_2 f(x,\f(x))}
\]
\end{tet}
\subsection{Igaz-e a következő állítás? \textit{"Az implicitfüggvény-tétel egy explicit előállítást ad az $f(x, y) = 0$ egyenlet implicit megoldására."} A válaszát indokolja meg!}
Nem igaz

Világos, hogy $\f(a) = b$. A $\f$ függvényt az $f(x, \f(x)) = 0 \; \; (x \in U)$ egyenlőség "implicit" módon definiálja. Innen származik a tétel neve.
A tétel csak a $\f$ implicit függvény létezéséről szól, ezt a függvényt általában nem tudjuk explicit képlettel előállítani. Ennek ellenére a $\f'(x)$ deriváltat ki tudjuk számítani, ha ismerjük a $\f(x)$ függvényértéket.
\subsection{A deriválási szabályok alapján hogyan vezethető le az $f(x, \f(x)) = 0 \; (x \in U)$ egyenlőségből az implicit megoldás deriváltjára vonatkozó összefüggést az $U$ környezetben?}
\[
F(x) := f(x,\f(x)) \quad (x \in U)
\]
Mivel $\forall x \in U $  esetén $F(x) = 0$, ezért $F'(x) = 0$. Az összetett függvény deriválási szabálya szerint
\begin{align*}
0 = F'(x) = \partial_1 f (x, \f(x)) &\cdot 1 + \partial_2 f(x,\f(x)) \cdot \f'(x) \quad (x\in U) \\
\text{ezért } \forall x \in U \text{ pontban:} \\
\f'(x) &= - \frac{\partial_1 f (x,\f(x))}{\partial_2 f(x,\f(x))}
\end{align*}
\subsection{Mit jelent az, hogy egy $\R^n \rightarrow \R^n$ típusú függvény lokálisan invertálható?}
\begin{tet}[Lokális invertálhatóság.]
Legyen $I \subset \R$ nyílt intervallum és $f : I \rightarrow \R$.

T.f.h. $f \in C^1(I)$ és egy $a \in I$ pontban $f'(a) \neq 0$

Ekkor $f$ az $a$-ban lokálisan invertálható, azaz $\exists U := K(a)$ és $V:= f[U]$ nyílt halmaz, hogy az  $f_{\mid_U} : U \rightarrow V$ függvény bijekció, ezért invertálható. Az $f_{\mid_U}^{-1}$ lokális inverz folytonosan deriválható $V$-n, és
\[
(f^{-1})' (y) = \frac{1}{f'(f^{-1}(y))} \quad (y \in V)
\]
\end{tet}
\subsection{Igaz-e, hogy minden $\R^2 \rightarrow \R^2$ típusú, folytonos és lokálisan invertálható függvény globálisan is invertálható? A válaszát indokolja meg!}
Nem igaz\\
Például az 
\[
f(x,y) := (e^x \cos y, e^x \sin y) \quad ((x,y)\in \R^2)
\]
folytonos függvény a sík minden $\pi$-nél kisebb sugarú körlapján injektív, de globálisan nem injektív, hiszen
\[
f(x,y+2\pi) = f(x,y) \quad (\forall (x,y)\in \R^2)
\]
\subsection{Hogyan szól az inverzfüggvény-tétel?}
\begin{tet}[Inverzfüggvény-tétel.]
Legyen $\Omega \subset \R^n \; (x\in\N)$  nyílt halmaz és $f: \Omega \rightarrow \R^n$. T.f.h
\begin{enumerate}
\item[(a)] $f$ folytonosan deriválható $\Omega$-n
\item[(b)] egy $a \in \Omega$ pontban $\det f'(a) \neq 0$
\end{enumerate}
Ekkor
\begin{enumerate}
\item $f$ lokálisan invertálható, azaz van olyan, az $a \in \Omega$ pontot tartalmazó $U$ nyílt halmaz, hogy ha $V := f[U]$, akkor az $f_{\mid_U} : U \rightarrow V$ függvény bijekció (következésképpen invertálható).
\item Az $\bb{f_{\mid_U}}^{-1}$ 
lokális inverz folytonosan deriválható $V$-n és
\[
\bb{f^{-1}}'(y) = \left[ f'(f^{-1}(y)) \right]^{-1} \quad (y \in V)
\]
\end{enumerate}
\end{tet}
\subsection{Igaz-e a következő állítás? \textit{"Az inverzfüggvény-tétel egy explicit előállítást ad bizonyos feltételeket teljesítő függvények inverzére."} A válaszát indokolja meg!}
Nem igaz

Az $f$ függvény explicit alakjának az ismeretében $f^{-1}$ helyettesítési értékeire általában nincs explicit képlet, viszont
\[
\bb{f^{-1}}'(y) = \left[ f'(f^{-1}(y)) \right]^{-1} \quad (y \in V)
\]
alapján a derivált helyettesítési értékei az $f'$ helyettesítési értékeinek felhasználásával már kiszámíthatók, ha ismerjük az inverz függvény értékét a megfelelő pontban
\newpage
\section{week}
\subsection{ Adja meg a két változós valós értékű $f$ függvény a $g = 0$-ra vonatkozó feltételes abszolút maximumának a fogalmát!}
\begin{defi}
Legyen $U \subset \R^2$ nyílt halmaz T.f.h $f,g: U \rightarrow \R $ adott függvények és
\[
H_g := \braces{z\in U \mid g(z) = 0} \neq 0
\]
$a \in H_g$ pontban \textbf{feltételes abszolút maximuma van} ha
\[
f(x) \leq f(a), \quad \forall x \in H_g \subset \CD_f = U
\]
\end{defi}

\subsection{Adja meg a két változós valós értékű $f$ függvény a $g = 0$-ra vonatkozó feltételes lokális maximumának a fogalmát}
\begin{defi}
Legyen $U \subset \R^2$ nyílt halmaz T.f.h $f,g: U \rightarrow \R $ adott függvények és
\[
H_g := \braces{z\in U \mid g(z) = 0} \neq 0
\]
$a \in H_g$ pontban \textbf{feltételes lokális maximuma van} ha
\[
\exists K(a) \subset U : f(x) \leq f(a), \quad \forall x \in K(a) \cap H_g 
\]
\end{defi}

\subsection{ Igaz-e, hogy egy feltételes abszolút maximum egyben feltételes lokális maximum? A válaszát indokolja meg!}
Igen
Mert van egy környezet amiben lokális maximum lesz

\subsection{Mondja ki az elsőrendű szükséges feltételről szóló tételt feltételes lokális szélsőértékekre!}
\textbf{Általános eset:}
\begin{tet}
T.f.h $n,m \in \N \; m < n, \; \emptyset \neq U \subset \R^n$ nyílt halmaz
\begin{enumerate}
\item[(a)] az $f : U \rightarrow \R $ és a $g = (g_1,\ldots , g_m) : U \rightarrow \R^m$ függvények folytonosan deriválhatók az $U$ halmazon
\item[(b)] az $a = (a_1, \ldots, a_n) \in U$ pontban az $f$ függvénynek a $g_1 = 0,\ldots ,g_m = 0$ feltételekre vonatkozóan feltételes lokális szélsőértéke van
\item[(c)] rang $\begin{bmatrix}
\partial_1 g_1(a) & \partial_2 g_1(a) & \ldots & \partial_n g_1(a) \\
\vdots & \vdots & \vdots & \vdots \\
\partial_1 g_m(a) & \partial_2 g_m(a) & \ldots & \partial_n g_m(a) 
\end{bmatrix} = 0 $
\end{enumerate}
Ekkor léteznek olyan $\lambda_1 , \ldots , \lambda_m \in \R$ (Lagrange-szorzók), hogy az
\[
\CL(x) := f(x) + \lambda_1 g_1 (x) + \ldots + \lambda_m g_m(x) \quad (x\in U)
\]
Lagrange-függvénynek $a = (a_1, \ldots , a_n)$ stacionárius pontja, azaz
\[
\CL' (a) = \boxes{\partial_1\CL(a), \ldots, \partial_n\CL(a)} = \boxes{0, \ldots ,0} = \theta_m \in \R^n
\]
\end{tet}

vagy 
\textbf{2 változos eset:}
\begin{tet}
T.f.h
\begin{enumerate}
\item[(a)] $U \subset \R^2 $ nyílt halmaz és az $f,g : U \to \R $ függvényeknek léteznek a parciális deriváltjaik és azok folytonosak az $U$ halmazon
\item[(b)] az $(x_0,y_0) \in U $ pontban az $f$ függvénynek a $g = 0$ feltételre vonatkozóan feltételes lokális szélsőértéke van
\item[(c)] $g' (x_0,y_0) = \bb{\partial_1 g(x_0,y_0),\partial_2 g(x_0,y_0)} \neq (0,0)$
\end{enumerate}
Ekkor van olyan $\lambda \in \R$ valós szám (ezt Lagrange-szorzónak
szokás nevezni), hogy az
\[
\CL(x,y) := f(x,y) + \lambda g(x,y) \quad \bb((x,y)\in U)
\]
Lagrange-függvénynek $(x_0, y_0)$ stacionárius pontja, azaz
\[
\CL'(x,y) = \bb{\partial_x\CL(x_0,y_0),\partial_y\CL(x_0,y_0)} = (0,0)
\]
\end{tet}
\subsection{Mondja ki a másodrendű elégséges feltételről szóló tételt feltételes lokális szélsőértékekre!}
\textbf{Általános eset:}
\begin{tet}
$n,m \in \N \; m < n, \; \emptyset \neq U \subset \R^n$ nyílt halmaz T.f.h:

$f,g_1,\ldots,g_m \in C^2$  és $\lambda_1,\ldots, \lambda_m \in \R$ olyan számok, valamint az $a \in U$ olyan pont, hogy az
\[
\CL := f+ \lambda_1g_1 + \ldots + \lambda_mg_m
\]
függvényre $\CL'(a) = \emptyset_n$ továbbá minden olyan $h \in \R^n, h \neq \subset_n$ vektorra, amelyre
\[
g_1'(a) \cdot h = 0, \; g_2'(a) \cdot h = 0, \ldots , \; g_m'(a) \cdot h = 0
\]
teljesül úgy, hogy 
\[
\angels{\CL''(a) \cdot h,h} > 0
\]
Ekkor az $f$ függvénynek a $g_1 = 0, \ldots , g_m = 0$ feltételek
mellett feltételes minimuma van az $a \in U $ pontban.
\end{tet}

vagy 
\textbf{2 változos eset:}
\begin{tet}
T.f.h
\begin{enumerate}
\item[(a)] $U \subset \R^2 $ nyílt halmaz és $f,g \in C^2(U,\R)$
\item[(b)] az $(x_0,y_0) \in U  $ pontban a $\lambda_0 \in \R $ számmal teljesül a szükséges feltétel. 
\end{enumerate}
Tekintsük ezzel a $\lambda_0$ számmal az
\[
\CL(x,y):= f(x,y) + \lambda_0g(x,y) \quad \bb{(x,y) \in U}
\]
Lagrange-függvényt. Legyen
\[
D(x_0,y_0;\lambda_0) := \det\begin{bmatrix}
0 & \partial_1g(x_0,y_0) & \partial_2g(x_0,y_0) \\
\partial_1g(x_0,y_0) & \partial_{11}\CL(x_0,y_0) & \partial_{12}\CL(x_0,y_0) \\
\partial_2g(x_0,y_0) & \partial_{21}\CL(x_0,y_0) & \partial_{22}\CL(x_0,y_0)
\end{bmatrix}
\]
(a mátrixot \textbf{kibővített Hesse-mátrixnak} szokás nevezni).
Ekkor:
\begin{enumerate}
\item ha $D(x_0,y_0;\lambda_0)> 0 \Leftarrow (x_0,y_0)$ feltételes lokális \textbf{maximumhely}
\item ha $D(x_0,y_0;\lambda_0)< 0 \Leftarrow (x_0,y_0)$ feltételes lokális \textbf{minimumhely}
\end{enumerate}
\end{tet}


\subsection{ Miért nem tudjuk általában alkalmazni a korábban megismert (nem feltételes) lokális szélsőértékek keresésére szolgáló módszert feltételes lokális szélsőértékek keresésére?}
Mert mindig feltettük, hogy a vizsgált pont az
értelmezési tartomány belső pontja. Könnyen látható azonban,
hogy a $H_g$ halmaznak nincs belső pontja.
\subsection{Milyen esetben tudjuk a kétváltozós függvényekre vonatkozó feltételes szélsőérték-problémát visszavezetni egy egyváltozós függvény szélsőérték-problémájára?}
T.f.h a feltételt megadó $g(x,y) = 0$ egyenletből (például) az $y$
kifejezhető az $x$ változó függvényeként, azaz $\exists \f \in \R \to \R $ függvény, amelyre $g(x,\f(x)) = 0 $

A $H_g = \braces{(x,y) \mid g(x,y) = 0} \subset \R^2$ halmaz tehát a $\f$ függvény garfikonja, ami "jó" esetben egy síkbeli "görbe". Az $f$ függvénynek a $H_g$ halmaz pontjaiban felvett értékeit a $h(x) := f(x,\f(x))$ alós-valós függvénnyel lehet kifejezni.

A kétváltozós függvényekre vonatkozó feltételes szélsőérték-problémát a szóban forgó esetben a $h$ egyváltozós függvény szélsőérték-problémájára lehet visszavezetni.
\subsection{Milyen esetekben és hogyan tudjuk a Weierstrass-tételt alkalmazni a feltételes abszolút szélsőrtékek keresésében?}
A feltételes abszolút szélsőértékhelyek megkeresése
egy "egyszerűbb" feladathoz vezethet, ha a
\[
H_g := \braces{x\in U \mid g(x) = 0}
\]
halmaz korlátos és zárt. Ebben az esetben a Weierstrass-tétel
garantálja a feltételes abszolút szélsőértékhelyek létezését, amelyek a Lagrange-függvény stacionárius pontjai lesznek.
\newpage
\section{week}
\subsection{Mit nevezünk szakaszonként sima útnak?}
\begin{defi}
A $\f : \boxes{a,b} \to \R^n$ függvény \textbf{szakaszonként sima
út}, ha
\begin{itemize}
\item $\f \in C \boxes{a,b}$
\item $\exists a = t_0 < t_ 1 < \ldots < t_m = b \; (m \in \N )$  olyan felosztása $\boxes{a,b}$-nek , amelyre tetszőleges $k = 0,1,\ldots,m-1$ index esetén $\f_{\vert_{\boxes{t_k,t_{k+1}}}}$ sima út
\end{itemize}
\end{defi}
\subsection{Mit nevezünk egy út ellentettjének?}
Egy $\f$ út $\tilde{\f}$ \textbf{ellentettjét} így definiáljuk:
\[
\tilde{\f} := \f(b+a-t) \quad (a \leq t \leq b)
\]
\subsection{Mit nevezünk az $u$ és $v$ pontokat összekötő szakasznak?}
Legyenek adottak az $u,v \in \R^n$  pontok, és legyen
\[
\f_{uv}(t) := u+t(v-u) \quad (0 \leq t \leq 1)
\]
Ekkor $\f_{uv}$ egy sima út, az $u$-t és $v$-t összekötő szakasz, amelynek a $\f_{uv}(0) = u$ a kezdőpontja, a $\f_{uv}(1) = v$ pedig a végpontja.
\subsection{Mikor mondjuk, hogy egy halmaz összefüggő, és mit nevezünk tartománynak?}
Azt mondjuk, hogy az $U \subset \R^n$ nyílt halmaz \textbf{összefüggő}, ha bármely két pontja összeköthető $U$-ban haladó töröttvonallal. Az összefüggő nyílt halmazokat röviden \textbf{tartománynak} nevezzük.
\subsection{Adja meg az $f$ függvény $\f$ útra vett vonalintegráljának fogalmát!}
\begin{defi}
T.f.h $U \subset \R^n \quad (n \in \R^n)$ tartomány, az $f: U \to  \R^n$ függvény folytonos, továbbá $\f : \boxes{a,b} \to  \R^n$ egy $U$-ban  haladó szakaszonként sima út. Ekkor az $f$ függvény $\f$ út vett vonalintegrálját így értelmezzük:
\[
\int\limits_\f f := \int\limits^b_a \angels{f \circ \f,\f'} = \int\limits^b_a \angels{f(\f(t)), \f'(t)} dt.
\]
\end{defi}
\subsection{Mondja ki a vonalintegrál utak egyesítéséről szóló állítás!}
\begin{tet}
Legyen $U \subset \R^n \quad (n\in \N)$ egy tartomány és t.f.h az $f,g : U \to \R^n$ függvények folytonosak.
Ha a $\f,\psi$ utak $U$-beliek és létezik a $\f \vee \psi $ egyesítésük, akkor
\[
\int\limits_{\f \vee \psi } f = \int\limits_\f f + \int\limits_\psi f
\]
\end{tet}
\subsection{Mondja ki a vonalintegrál utak ellentettjéről szóló állítás!}
\begin{tet}
Legyen $U \subset \R^n \quad (n\in \N)$ egy tartomány és t.f.h az $f,g : U \to \R^n$ függvények folytonosak.
bármilyen $U$-beli $\f$ út $\tilde{\f}$ ellentettjére
\[
\int\limits_{\tilde{\f}} f = -\int\limits_\f f 
\]
\end{tet}
\subsection{Adja meg egy $f$ vektormező primitív függvényének fogalmát!}
\begin{defi}
Legyen $U \subset \R^n$ egy tartomány és  $f = (f_1,\ldots,\f_n): U \to \R^n$ adott vektormező Azt mondjuk, hogy a $F : U \to \R$ függvény a $f$ függvény \textbf{primitív függvénye} ha $F$ differenciálható $U$-ban, és $F' = f$ azaz  ha minden $x \in U$ pontban
\[
F'(x) = \bb{\partial_1F(x),\ldots,\partial_nF(x)} = \bb{f_1(x),\ldots,f_n(x)}
\]
\end{defi}
\subsection{Mondja ki a Newton-Leibniz-tételt!}
\begin{tet}[Newton–Leibniz]
Legye $U \subset \R^n$ egy tartomány, és t.f.h. az $f : U \to \R^n$
folytonos függvénynek van primitív függvénye. Ekkor tetszőleges $U$-ban haladó $\f : [a, b] \to U$ szakaszonként sima út esetén a $f$ bármelyik $F$ primitív függvényével
\[
\int\limits_\f f = F(\f(b))-F(\f(a))
\]
\end{tet}
\subsection{Igaz-e a következő állítás? "Ha a folytonos $f : U \to \R^n$ függvénynek van primitív függvénye, akkor $f$ vonalintegráltjának értéke nulla tetszőleges $U$-ban haladó zárt úton" A válaszát indokolja meg!}
\newpage
\section{week}
\subsection{Mit jelent, hogy egy vonalintegrál független az úttól?}
\begin{tet}
Legyen $U \subset \R^n$ tartomány és $f = (f_1,\ldots,f_n): U \to \R^n$ folytonos függvény.

A vonalintegrál független az úttól. Ez azt jelenti, hogy ha az $U$-be
\[
\f : \boxes{a,b} \to U \quad \text{ és } \quad \psi : \boxes{c,d} \to U
\]
szakaszonként sima utak $\f(a) = \psi(c)$ és $\f(b) = \psi(d)$ azaz a $\f,\psi$ utak ugyanazt a kezdőpontot és végpontot köti össze $U$-ban, akkor
\[
\int\limits_\f f = \int\limits_\psi f
\]
\end{tet}
\subsection{Milyen állításokat ismer, amelyek ekvivalensek azzal, hogy minden vonalintegrál független az úttól?}
\begin{tet}
Legyen $U \subset \R^n$ tartomány és $f = (f_1,\ldots,f_n): U \to \R^n$ folytonos függvény.
\begin{enumerate}
\item A $f$-nek létezik primitív függvénye $U$-n, vagyis $\exists F : U \to \R$ differenciálható függvény, amelyre minden $x \in U$ pontban
\[
F'(x) = \bb{\partial_1F(x),\ldots,\partial_nF(x)} = \bb{f_(x),\ldots,f_n(x)}
\]
\item Minden $U$-ban haladó $\f : \boxes{a,b} \to U $ U szakaszonként sima zárt (az $\f(a) = \f(b)$) útra
\[
\oint\limits_\f f = 0
\]
\end{enumerate}
\end{tet}
\subsection{Mondja ki a tanult szükséges feltételt primitív függvény létezésére vonatkozóan!}
\begin{tet}[Szükséges feltétel primitív függvény létezésére]
Legye $U \subset \R^n$ tartomány és $f = \bb{f_1,\ldots,f_n} : U \to \R^n$ deriválható függvény. Ha $f$-nek létezik primitív függvénye $U$-n, akkor az $f'$ deriváltmátrix szimmetrikus, azaz minden $x \in U$ pontban
\[
\partial_if_j(x)=\partial_jf_i(x) \quad \bb{i,j = 1,2,\ldots,n}
\]
\end{tet}
\subsection{Igaz-e a következő állítás? "Minden $f : \R^2 \to \R^2$ folytonos függvénynek van primitív függvénye." A válaszát indokolja meg!}
Nem igaz
Például
\[
f(x,y) := \bb{-\frac{y}{x^2+y^2},\frac{x}{x^2+y^2}} \quad \bb{(0,0) \neq (x,y) \in \R^2}
\]
Ennek a függvénynek $\int\limits_\f f \neq 0$ Mivel $\f$ zárt út $\mathcal{D}_f$-ben ezért $f$-nek nincs primitív függvénye. Az az deriválható még sincs primitív függvénye
\subsection{Mondja ki a tanult elégséges feltételt primitív függvény létezésére vonatkozóan!}
\begin{tet}[Elégséges feltétel primitív függvény létezésére]
Tekintsük az $U \subset \R^n \quad (n\in \N)$ csillagtartományon értelmezett $f=\bb{f_1,\ldots,\f_n} : U \to \R^n$ folytonosan deriválható függvényt.
T.f.h $\forall x \in U$ esetén az $f'(x)$ deriváltmátrix szimmetrikus, azaz minden $x \in U$ pontban
\[
\partial_if_j(x)=\partial_jf_i(x) \quad \bb{i,j = 1,2,\ldots,n}
\]
Ekkor $f$-nek van primitív függvénye, azaz $\exists F: U \to \R$ differenciálható függvény, hogy $\forall i = 1,\ldots,n$ index esetén $\forall x \in U$ pontban $\partial_iF(x)=f_i(x)$ 
\end{tet}
\subsection{Adja meg egy $v$ vektormező divergenciájának fogalmát!}
\begin{defi}
A $v = \bb{v_1,v_2,v_3} : D \to \R^3 \; \bb{D \subset \R^3 \text{ tartomány} }$ deriválható vektormező $v'$ deriváltmátrixának főátlójában álló elemeinek összegét, azaz a
\[
\text{div } \textbf{v} := \partial_1v_1+\partial_2v_2 + \partial_3v_3 : D \to \R
\]
függvényt a $v$ vektormező \textbf{divergenciájának} nevezzük.
\end{defi}
\subsection{Adja meg egy $v$ vektormező rotációjának fogalmát!}
\begin{defi}
A $v = \bb{v_1,v_2,v_3} : D \to \R^3 \; \bb{ D \subset \R^3  \text{ tartomány} }$ deriválható vektormező \textbf{rotációjának} a 
\[
\text{rot } \textbf{v}:=\boxes{\partial_2V_3-\partial_3V_2 \;\; \partial_3V_1-\partial_1V_3 \;\; \partial_1V_2-\partial_2V_1}
\]
függvényt nevezzük.
\end{defi}
\subsection{Mondja ki a Green-tételt!}
\begin{tet}[Green-tétel]
T.f.h $\f : \boxes{0,1} \to \R^2$ pozitív irányítású, szakaszonként sima, egyszerű, zárt görbe, és $S \subset \R^2$ az általa határolt síkrész.
Legye $f \in \R^2 \to \R^2, S \subset \CD_f$ folytonosan differenciálható függvény. Ekkor
\[
\int\limits_{\partial S} f = \iint\limits_s (\partial_1f_2-\partial_2f_1)
\]
ahol $\partial S$ az $S$ határát jelöli és $\f$ a $\partial S$ egy paraméterezése.
\end{tet}
\end{document}