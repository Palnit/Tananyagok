\documentclass[10pt,a4paper]{article}
\usepackage[utf8]{inputenc}
\usepackage[magyar]{babel}
\usepackage[T1]{fontenc}
\usepackage{amsmath}
\usepackage{amsfonts}
\usepackage{amssymb}
\usepackage{makeidx}
\usepackage{graphicx}
\usepackage[pdfusetitle]{hyperref}
\usepackage{titling}

\hypersetup{
    colorlinks,
    citecolor=black,
    filecolor=black,
    linkcolor=black,
    urlcolor=black
}

\date{\today}
\author{Petrányi Bálint}
\title{%
	\textbf{Analízis 2.} \\
	\textbf{Programtervező informatikus A. szakirány} \\
	\medskip
	Bizonyítások\\
	\large 2022-2023. tanév 1. félév
}

\begin{document}
\maketitle
\tableofcontents
\newpage
\section{A differenciálhatóság átfogalmazása}
\textbf{Tétel:} \\ 
Legyen $f \in \mathbb{R} \rightarrow \mathbb{R}, a \in \text{int } \mathcal{D}_f$. Ekkor
\[
\begin{split}
f \in & \; D\{a\} \\
& \Updownarrow \\
\exists A \in \mathbb{R}, \; \text{és} \; \epsilon : \mathcal{D}_f \rightarrow \mathbb{R}, \lim_a{\epsilon} = 0 \quad \text{úgy, hogy } & \; f(x)- f(a) = A(x-a)+\epsilon (x)(x-a)
\end{split}
\]
\textbf{Bizonyítás:}
\[
\Longrightarrow f \in D\{a\} \; \text{esetén legyen} \; A:=f'(a), \; \text{és} \epsilon (x) := \frac{f(x)-f(a)}{x-a} -f'(a).
\]
Ezzel a választással egyrészt
\[
A(x-a) + \epsilon (x)(x-a) =f'(a)(x-a)+\Big(\frac{f(x)-f(a)}{x-a}-f'(a)\Big)(x-a) =f(x)-f(a),
\]
másrészt a differenciálhatóság miatt
\[
\lim\limits_{x\rightarrow a} \epsilon (x) = \lim\limits_{x\rightarrow a} \frac{f(x)-f(a)}{x-a}-f'(a)=0.
\]
$\Longleftarrow$ Ha az $A$ szám és az $\epsilon$ függvény teljesítik az állítást feltételeit, akkor 
\[
\frac{f(x)-f(a)}{x-a} = A + \epsilon (x), és \lim\limits_{x-a} \frac{f(x)-f(a)}{x-a} = A = f'(a) \in \mathbb{R}
\]
\newpage
\section{A folytonosság és a differenciálhatóság kapcsolata}
\textbf{Tétel:}
\[
f \in \mathbb{R} \rightarrow \mathbb{R}, \quad f \in D\{a\} \quad \Longrightarrow \quad  f \in C\{a\}
\]
\textbf{Szóban:} Ha egy valós-valós függvény differenciálható egy pontban, akkor folytonos abban a pontban. \\ \\
\textit{Megjegyzés:} Fordítva nem igaz. Az abszolútérték függvény folytonos a $0$ pontban de nem deriválható $0$-ban. Kompatibilitási probléma is van: folytonosság esetében nem szükséges, hogy a pont az értelmezési tartomány belső pontja legye. \\ \\
\textbf{Bizonyítás:} \\ \\
Nyilván $f(x)=\frac{f(x)-f(a)}{x-a}(x-a)+f(a). \; f$ differenciálhatósága miatt, ezért
\[
\lim_a f= \lim_{x\rightarrow a } \Big(\frac{f(x)-f(a)}{x-a}(x-a)+f(a)\Big) 	= f'(a)\cdot 0 + f(a)=f(a)
\]
\newpage
\section{Deriválási szabályok}
\textbf{Tételek:} \\ \\
$f,g \in \mathbb{R} \rightarrow \mathbb{R}$ \\

\begin{tabular}{c c c c c c c c}
1. & $ f\in D\{a\} \text{ és } c \in \mathbb{R} \:$ & $\Longrightarrow$& $ c\cdot f \in D\{a\}, $ & $\text{és}$ & $ (c \cdot f)'(a) $&$=$&$c\cdot f'(a) $ \\
2. & $f\in D\{a\} $&$\Longrightarrow$ &$ f+g\in D\{a\},$ &$ \text{és}$&$ (f+g)'(a) $&$=$&$f'(a) + g'(a)$   \\
3. & $f\in D\{a\}$ & $\Longrightarrow $ & $ f\cdot g \in D\{a\}, $&$  \text{és} $&$ (f\cdot g)'(a) $&$=$&$f'(a)g(a)+ f(a)g'(a)$ \\
4. & $f\in D\{a\} \text{ és }  g(a)\neq 0 $&$\Longrightarrow$&$  \frac{f}{g} \in D\{a\},$&$  \text{és} $&$ \Big(\frac{f}{g} \Big)' (a)$&$ =$&$ \frac{f'(a)g(a)-f(a)g'(a)}{g^2(a)}$
\end{tabular}
\end{document}